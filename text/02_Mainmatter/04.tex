\chapter{Methodology}
\label{chap:four}

\section{\acl{ML}}

It is generally agreed upon that the term 
\ac{ML} refers to the field of study that gives computers the ability to learn 
without being explicitly programmed. This fact was first 
introduced by Arthur Samuel in 1959 \cite{Samuel1959}. However, note that the reference to this paper is used
loosely, as the term \ac{ML} was not directly used in the paper and it is rather 
a retrospective interpretation of the paper. 

The term \ac{ML} was more explicitly introduced by Tom M. Mitchell in 1997 \cite{Mitchell1997}:

\begin{defin}[Machine Learning]\label{de:ml}
    A computer program is said to learn from experience E with respect
    to some class of tasks T and performance measure P, if its performance at tasks in
    T, as measured by P, improves with experience E. 
\end{defin}

Typically, the experience E is represented by a dataset, which is used to train the model. 
We can generally say that \ac{ML} is the ability to get better at specified task by learning
from provided relevant data without problem domain specific programming of the computer.


Nowadays \ac{ML} is at the core of many applications that we use daily. The usecases range from
spam filters, recommendation systems, medical diagnosis, stock trading, and many more. Currently, 
machine learning is dominated by deep learning, which is a subfield of \ac{ML} that focuses on
deep neural networks that rely on large datasets in order to be able to generalize well.
On the other, hand traditional \ac{ML} algorithms are still widely used and are often the first choice
when the dataset is small or the problem is low dimensional. Whereas deep learning models
can be used in high frequency data, where the datasets are large enough to train the model, 
daily closing stock price prediction is a task that can be successfully solved with 
traditional \ac{ML} algorithms. As the datasets are much smaller. 


In general, cryptocurrencies
lie somewhere in the middle of the spectrum. Exactly as stocks, either high frequency or low frequency
data can be chosen based on the research question. The difference however is that cryptocurrencies 
are much more volatile than stocks, and they can have much higher dimensionality as we can use 
many technical analysis indicators to predict the price. 
That is why we will use a combination of
traditional \ac{ML} algorithms and deep learning models to predict the 
closing prices or returns of
various cryptocurrencies.

\ac{ML} algorithms can be divided into three main categories: supervised learning, unsupervised learning, and reinforcement learning.
As mentioned earlier, we will focus on supervised learning as the process of forecasting
can be easily transformed into a supervised learning problem where the input features
are historical or current data and the output is the future price or return.


It is important to note on the fundamental difference between machine learning and
traditional econometric models. In econometrics the focus is to uncover the underlying
relationship between the variables and to understand the size of contribution of each
feature. In machine learning, the focus is mainly on maximizing the performance metric for predictions and 
the magnitude of the effects usually remains unknown. This is definitely a weakness of machine learning
which researchers try to adress by developing new field of explainable \ac{AI}. 
Where they focus on developing models that are able to explain their predictions in a human understandable way.
They provide a significant promise for the use of \ac{ML} methods in finance in the future
where the interpretability of the model is crucial for customers or regulators in specific subfields.





\section{Ridge Linear Regression}

\section{Support Vector Machines}

\section{\acl{LSTM} \acl{RNN}s}

\section{\acl{PCA}}

\section{Stationarity in Time-Series}

We would like to explicitly adress the concept of stationarity as there 
seems to be a lot of confusion around the topic especially because the boundary between 
traditional econometric models and machine learning models is not always clear.

Stationarity is a crucial concept in time-series analysis. Despite the fact, that there exist
many tests and rigid definitions for it, it is often reduced to the 
concept of constant mean and variance or generally to the fact that the parameters
of the distribution do not change in time. Without this property the errors
of the model become function of time which is not desirable for proper inference.

Econometrics usually prefers stationary data for aforementioned reasons because it 
is cruicial for inference and interpretation of the results. That is 
why econometricians like to model returns instead of prices, as returns
should generally be stationary. Imporantly, there is always the possibility
to transform the predictions back to the original prices by adding the forecasted
returns to the last observed price and potentially reversing some normalization steps.

This is different for \ac{ML} as the focus is on prediction and the models are fundamentally 
different. As the models are trained using gradient descent and not using the
analytical solutions, the stationarity is not as crucial. Clearly, 
the models will perform better on stationary data because it requiers much 
less capacity for the model as some of the information removed by differencing.
However, models with enough capacity can easily learn non-stationary data and
capture information about trends, seasonalities, and other patterns. 
Thus we decided to test our models on both prices directly and on returns where
the results are much less dependent on changes in the distribution of the data.

\section{Proposed Forecasting Framework}

We propose a forecasting framework which is designed to compare the performance between using \ac{PCA} as a 
dimensionality reduction technique and using the raw data. 
The framework is shown in Figure \ref{fig:forecasting_framework}.
The preprocessing layer is responsible for cleaning the data, filling in missing values and tranforming 
the problem to supervised learning as described 
in \ref{chap:three}. Following stage is responsible for reducing the number of features. 
The first \ac{PCA} step transforms the data onto \textit{n} principal components and the filtering step
chooses the most important features such that their cumulative variance adds up to 95\%, 98\% or 99\%.
Following is the \ac{LSTM} reshaping layer that is responsible for transforming the data into a 3D tensor for 
the \ac{LSTM} \ac{RNN} as described 
in \ref{chap:three}. The most crucial layer is the forecasting layer that essentially consists of 3 parts.
Firstly, it normalizes the variables using robust scaler to ensure that the input features come from 
the same value range. Secondly, it uses grid search to search for the best
hyperparameters for the model. This is a cruical step as the change of dimensionality changes 
the optimal hyperparameters. Without this step it would be hard to establish any real
effects as the change in performance might be attributed to the change in suitability of hyperparameters.
Lastly, it trains the model and evaluates the performance using the best found model. This is the reason
why we abstract the metrics layer seperately to make it apparent that the metrics are calculated
on the best model found by the grid search. The metrics layer is also the layer where
we can statistically compare the significance of the difference between the models.
It is important to note that this framework is executed across all specified forecasting models, 
cryptocurrencies of interest, and forecasting horizons as we expect the effects
to be quite different for different models and forecasting horizons. 


As we have already noted the fact that that the target price is clearly non-stationary,
has important implication for the choice of splitting strategy for the grid search 
hyperparameter optimization. 
There are two main strategies for splitting the data into training and testing sets.
The first one is traditional train, development, and test split where the model is trained on the training set,
hyperparameters are optimized on the development set, and the model is evaluated on the test set.
This approach is generally sufficient when the dataset is large enough and we believe
that the distributions of the three datasets are similar. 
A more robust approach is to use grid search with cross-validation. 
This approach makes sure that we have not overfitted the model to the development set.
Firstly, we split the data into training and testing set. And then 
we iteratively split the training set into training and validation set. 
Train the model on the training part of the train set and evaluate on the validation set.
Finally we average over the results across all validation sets and take the best hyperparameters
from the provided grid. This approach avoids overfitting and is typically
used when we have enough compute to train the models. 


Note that the goal of grid search is to create a robust representation of a 
reasonable test set distribution as it samples randomly the points into the training and validation set.
This approach works if the data is stationary and the distribution of the data does not change drastically in time.
But time series rarely have this property and cryptocurrencies suffer especially profound changes in the distribution
as they have grown in popularity. The difference between the training and test set
is extreme but that is something that we can do very little about as we want the test set
to be as close to the future as possible. 


Additionally, there is an extremely
common mistake that researchers make when they use grid search for time series data.
This problem is called data leakage and it occurs when future distribution of the data
is used to optimize the hyperparameters. As \ac{ML} models typically 
require data to be scaled and normalized, the parameters for the normalization 
need to be learned only on the historical data not on the future data otherwise the 
results will be biased and suggest higher quality of the model than it actually is.
Theoretically, this could be relaxed on the training set as the normalization
might be fitted on the whole training set but it is crucial that the test set distribution
is never leaked to the fitting procedure. However, we believe that
the correct methodology is to use a rolling window approach for the grid search
where the normalization parameters are fitted on the training set and the validation set.
This is why we use time series split for the grid search without randomly shuffling the data.
This approach splits the training set into \textit{k} folds and iteratively
trains the model on \textit{k-1} folds and evaluates on the \textit{k-th} fold.
This approach has the advantage that the model is actually trained with
different sized training sets which is useful for generalization properties
but has the disadvantage that the difficulty of the splits is extremely volatile.
This is definitely a limitation as the performances between splits
can vary significantly and averaging over the results might be prone to noise.
However, we believe that this is the methodologically correct approach.
\begin{figure}[!h]
    \centering
    \caption{Time Series Split with \textit{k=2}} 
        \includegraphics[width=1\textwidth]{Figures/time_series_split.drawio.pdf}
    \label{fig:ts_split}
\end{figure}
We opted to use only two splits as the the data size
is relatively small and the results were extremely noisy when using
more splits.


\begin{figure}[!h]
    \centering
    \caption{Proposed Forecasting Framework}
        \includegraphics[width=1\textwidth]{Figures/Forecasting_framework.drawio.pdf}
    \label{fig:forecasting_framework}
\end{figure}