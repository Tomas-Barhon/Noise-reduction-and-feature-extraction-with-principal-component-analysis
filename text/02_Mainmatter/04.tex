\chapter{Methodology}
\label{chap:four}

\section{Title of Section One}

Many people use simple n-dash in many occasions -- like this --, where however typographic convention---it looks a bit strange at first sight---requires m-dash. Text text text text text text text text text text text text text text text. Text text text text text text text text text text. Text text text text text text \citet{Haufler2006}. 

Text text text text text text text text text text text text text text text. Text text text text text text text text text text. Text text text text text text. Text text text text text text text text text text text text text text text. Text text text text text text text text text text. Text text text text text text \citet{Wells2001}. Let us describe the following animals:

\begin{description}
\item[Item 1] Text text text text text text text text text text text text text text text. Text text text text text text text text text text. Text text text text text text. Text text text text text text text text text text text text text text text. Text text text text text text text text text text. Text text text text text text.
\item[Item 2] Text text text text text text text text text text text text text text text. Text text text text text text text text text text. Text text text text text text. Text text text text text text text text text text text text text text text. Text text text text text text text text text text. Text text text text text text.
\end{description}

Text text text text text text text text text text text text text text text. Text text text text text text text text text text. Text text text text text text. Text text text text text text text text text text text text text text text. Text text text text text text text text text text. Text text text text text text. See what Edmund Burke said about the duties of a Member of Parliament (Speech To The Electors Of Bristol At The Conclusion Of The Poll, November 3, 1774):

\begin{quotesmall}
It ought to be the happiness and glory of a representative to live in the strictest union, the closest correspondence, and the most unreserved communication with his constituents. Their wishes ought to have great weight with him; their opinion, high respect; their business, unremitted attention. It is his duty to sacrifice his repose, his pleasures, his satisfactions, to theirs; and above all, ever, and in all cases, to prefer their interest to his own. But his unbiased opinion, his mature judgment, his enlightened conscience, he ought not to sacrifice to you, to any man, or to any set of men living. These he does not derive from your pleasure; no, nor from the law and the constitution. They are a trust from Providence, for the abuse of which he is deeply answerable. Your representative owes you, not his industry only, but his judgment; and he betrays, instead of serving you, if he sacrifices it to your opinion.
\end{quotesmall}

Text text text text text text text text text text text text text text text. Text text text text text text text text text text. Text text text text text text.Text text text text text text text text text text text text text text text. Text text text text text text text text text text. Text text text text text text.

\begin{listi}
	\item The first item, the first item, the first item, the first item, the first item, the first item,
	\item and the second item.
\end{listi}

\begin{lista}
	\item The first item, the first item, the first item, the first item, the first item, the first item, 
	\item and the second item.
\end{lista}

Text text text text text text text text text text text text text text text. Text text text text text text text text text text. Text text text text text text. Text text text text text text text text text text text text text text text. Text text text text text text text text text text. Text text text text text text. Text text text text text text text text text text text text text text text. Text text text text text text text text text text. Text text text text text text \citet{Blomstrom2003}. 