\chapter{Literature Review}
\label{chap:two}


\section{Cryptocurrencies}
\label{sec:crypto}

\subsection{Bitcoin}
In the year 2008, an unknown author with the pseudonym Satoshi Nakamoto introduced the idea of 
a purely peer-to-peer electronic cash system. Interestingly the author mentions 
small casual transactions as something that the current model relying on third-party
financial institutions fails to deliver because of unavoidable transaction costs \cite{Nakamoto2008}. 
In contrast, from today's perspective, Bitcoin is a relatively slow medium for micro-transactions because 
technically the receiver has to always wait for a certain amount of blocks to be mined such that
it becomes statistically unlikely that double-spending has been committed by the payer 
\cite{Conti2018}. This phenomenon
can be demonstrated on the data from \textbf{\href{https://coinmetrics.io/}{coinmetrics.io}} which show
that the mean size of a \ac{BTC} transaction ranges in thousands of USD\$. 
Another important aspect is the fact that 
the miners prioritize transactions with higher fees in the block which introduces considerable costs
to each payment. \cite{Moeser2015} have shown the relationship between the transaction fee 
and the transaction latency meaning the time it takes for the transaction to be almost surely valid.
Even though 
there has been some divergence from the original idea of small transactions all of the security measures
regarding the double spending problem in the original whitepaper have turned out to be well-defined 
in a medium time horizon.


Despite the fact, that Bitcoin is generally regarded as the first cryptocurrency it relies on 
many older ideas and technologies that are mostly mentioned in the original whitepaper. First and foremost
stands the conference paper \textit{How to Time-Stamp a Digital Document} \cite{Haber1991} which focuses
on the problem of third parties responsible for verification of digital documents. It makes use of 
an already established family of functions known as hashes that surpass privacy concerns and generally
surpass the need for a third party to be involved in the verification process when combined
with the correct consensus algorithm. They define a hash function as follows:

\begin{defin}[Hash]\label{de:hash}
    This is a family of functions $h: \{0,1\} \rightarrow  {\{0, 1\}}^l $ compressing
bit-strings of arbitrary length to bit-strings of a fixed length l, with the following
properties:
\begin{enumerate}
    \item The functions $h$ are easy to compute, and it is easy to pick a member of the family at random.
    \item It is computationally infeasible, given one of these functions $h$, to find a pair
of distinct strings $x,x'$ satisfying $h(x)=h(x')$. (Such a pair is called a collision for $h$)\cite[see Chapter 4.1]{Haber1991}
\end{enumerate}
\end{defin}

Hashes are the most crucial building blocks of cryptocurrencies. However, the time-stamping might 
fail if the users can tweak the time of their machines. Interestingly, the authors have already 
mentioned that and introduced the idea of chaining the data together with their metadata sequentially in a long chain 
so that the user can trust that something was not overwritten \cite[see Chapter 5]{Haber1991}. 
This possibility is enabled by the properties of the hash functions.
Meaning that we can concatenate arbitrarily long inputs and always produce a fixed-size output.



Algo

Security


Research


\subsection{Ethereum}

\subsection{Litecoin}


\section{Machine Learning Methods for Cryptocurrencies}
\label{sec:ml}

\section{Principal Component Analysis}
\label{sec:pca}


\subsection{PCA in Other Areas}

\subsection{PCA in Time Series}


\section{Web Search Data in Financial Applications}





