\chapter{Literature Review}
\label{chap:two}


\section{Cryptocurrencies}
\label{sec:crypto}

\subsection{Bitcoin}
In the year 2008, an unknown author with the pseudonym Satoshi Nakamoto introduced the idea of 
a purely peer-to-peer electronic cash system. Interestingly the author mentions 
small casual transactions as something that the current model relying on third-party
financial institutions fails to deliver because of unavoidable transaction costs \cite{Nakamoto2008}. 
In contrast, from today's perspective, Bitcoin is a relatively slow medium for micro-transactions because 
technically the receiver has to always wait for a certain amount of blocks to be mined such that
it becomes statistically unlikely that double-spending has been committed by the payer 
\cite{Conti2018}. This phenomenon
can be demonstrated on the data from \textbf{\href{https://coinmetrics.io/}{coinmetrics.io}} which show
that the mean size of a \ac{BTC} transaction ranges in thousands of USD\$. 
Another important aspect is that 
the miners prioritize transactions with higher fees in the block which introduces considerable costs
to each payment. \cite{Moeser2015} have shown the relationship between the transaction fee 
and the transaction latency meaning the time it takes for the transaction to be almost surely valid.
Even though 
there has been some divergence from the original idea of small transactions all of the security measures
regarding the double spending problem in the original whitepaper have turned out to be relatively 
well-defined in a medium time horizon.


Despite the fact, that Bitcoin is generally regarded as the first cryptocurrency it relies on 
many older ideas and technologies that are mostly mentioned in the original whitepaper. First and foremost
stands the conference paper \textit{How to Time-Stamp a Digital Document} \cite{Haber1991} which focuses
on the problem of third parties responsible for verification of digital documents. It makes use of 
an already established family of functions known as hashes that surpass privacy concerns and generally
surpass the need for a third party to be involved in the verification process when combined
with the correct consensus algorithm. They define a hash function as follows:

\begin{defin}[Hash]\label{de:hash}
    This is a family of functions $h: \{0,1\} \rightarrow  {\{0, 1\}}^l $ compressing
bit-strings of arbitrary length to bit-strings of a fixed length l, with the following
properties:
\begin{enumerate}
    \item The functions $h$ are easy to compute, and it is easy to pick a member of the family at random.
    \item It is computationally infeasible, given one of these functions $h$, to find a pair
of distinct strings $x,x'$ satisfying $h(x)=h(x')$. (Such a pair is called a collision for $h$)\cite[see Chapter 4.1]{Haber1991}
\end{enumerate}
\end{defin}

And suggest hashing documents together with the time of their creation.
However, the time-stamping might 
fail if the users can tweak the time of their machines. Interestingly, the authors have already 
mentioned that and introduced the idea of chaining the data together with their metadata sequentially in a long chain 
so that the user can trust that something was not overwritten \cite[see Chapter 5]{Haber1991}. 
This is possible due to the properties of the hash functions.
Meaning that we can concatenate arbitrarily long inputs and always produce a fixed-size output. 
Another significant influence came from b-money which was an idea for an anonymous digital
cash system presented in \cite{Dai1998}. B-money proposed the concept of \ac{PoW}
which is a validation protocol that many cryptocurrencies still use. The idea was to solve
computationally challenging puzzles where it can be determined how much effort was used to do so
\cite[see][pg.~1]{Dai1998}.
However, certain worries were being raised about how to regulate a system if the computational power
of computers is increasing every year \cite[see][pg.~3]{Dai1998}. This has been addressed by Bitcoin
with the regulation of difficulty based on the average time it takes to solve the puzzle 
rather than the difficulty itself \cite[see][pg.~3]{Nakamoto2008}.


Since many characteristics of the network are often being used by researchers such as: 
(\cite{Kukacka2023}, \cite{Kristoufek2023}, \cite{Kubal2022} or \cite{Jay2020}) 
in their research, it is critical to understand the underlying mechanics that form them.
Bitcoin takes a completely adverse approach to the general financial system. Whereas traditionally
banks and other institutions try to keep every transaction encrypted Bitcoin makes all the 
transactions publicly visible and available but hashing the addresses of the sender and receiver.
The process of sending Bitcoins to someone else essentially means adding a digital signature to the
previous transaction from which you received that money. 
However, as \cite[see Chapter 2][pg.~2]{Nakamoto2008} suggests this only mitigates the privacy concerns
but the double spending risk needs to be dealt with a smarter design.
This is solved by the introduction of the \ac{PoW} algorithm. The idea is that
transactions are collected into blocks by the miners who try to solve a computationally difficult 
task that can only be solved by a brute-force search. It is simply a race to find a hash with a 
certain amount of leading zeros which is adjusted based on the mining power of the network.  
The miners are incentivized by \acs{BTC} price for winning the race and also by the transaction
fees that can be added to each transaction as a reward for being prioritized. 
The main concept is that the blocks are connected sequentially in a chain through the hash.
If we assume that most of the nodes/miners are honest their profit-maximizing behavior should always be
working on the longest chain and thus transactions that have already been spent do not get included
in the chain. After the puzzle is solved by a node it can be validated by all other nodes
in a linear time and they move on to the next block. Despite that, there remains the risk of 
an attacker forking a malicious block and sending his money back or elsewhere. \cite{Nakamoto2008}
claims that the probability of an attacker catching up (or reaching breakeven from
the memoryless property of poi) drops exponentially
with each block if the mining power (probability of solving the puzzle) of the
attacker is lower than the power of honest nodes. This mechanism is the root of the Bitcoin
security. However, it also implies that there is an implicit tradeoff between security and the
desired liquidity of cash. 


As cryptocurrencies are a relatively new phenomenon they are currently a frontier topic of academic 
research in many different aspects. They are being studied on multiple levels such as
law, technology, cryptography, security, economics or machine learning. Generally, our area
of interest lies in economics and machine learning as we want to find out
whether there exist some determinants of the bitcoin price or at least features that can be used 
to estimate the dynamics. We assume that there are theoretically three possibilities of the model 
of price of \ac{BTC} and other cryptocurrencies. Firstly, it might be a purely efficient market 
and the price technically follows a random walk process. The second model is that the price is entirely
driven by speculative components and lastly, it might be a combination of speculative and fundamental
components.

random walk/only speculative/also fundamental

\subsection{Ethereum}

\subsection{Litecoin}


\section{Machine Learning Methods for Cryptocurrencies}
\label{sec:ml}

\section{Principal Component Analysis}
\label{sec:pca}


\subsection{PCA in Other Areas}

\subsection{PCA in Time Series}


\section{Web Search Data in Financial Applications}





