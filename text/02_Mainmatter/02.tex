\chapter{Literature Review}
\label{chap:two}


\section{Cryptocurrencies}
\label{sec:crypto}

\subsection{Bitcoin}
In the year 2008, an unknown author with the pseudonym Satoshi Nakamoto introduced the idea of 
a purely peer-to-peer electronic cash system. Interestingly the author mentions 
small casual transactions as something that the current model relying on third-party
financial institutions fails to deliver because of unavoidable transaction costs \cite{Nakamoto2008}. 
In contrast, from today's perspective, Bitcoin is a relatively slow medium for micro-transactions because 
technically the receiver has to always wait for a certain amount of blocks to be mined such that
it becomes statistically unlikely that double-spending has been committed by the payer 
\cite{Conti2018}. This phenomenon
can be demonstrated on the data from \textbf{\href{https://coinmetrics.io/}{coinmetrics.io}} which show
that the mean size of a \ac{BTC} transaction ranges in thousands of USD\$. 
Another important aspect is the fact that 
the miners prioritize transactions with higher fees in the block which introduces considerable costs
to each payment. \cite{Moeser2015} have shown the relationship between the transaction fee 
and the transaction latency meaning the time it takes for the transaction to be almost surely valid.
Even though 
there has been some divergence from the original idea of small transactions all of the security measures
regarding the double spending problem in the original whitepaper have turned out to be well-defined 
in a medium time horizon.



Algo

Security


Research


\subsection{Ethereum}

\subsection{Litecoin}


\section{Machine Learning Methods for Cryptocurrencies}
\label{sec:ml}

\section{Principal Component Analysis}
\label{sec:pca}


\subsection{PCA in Other Areas}

\subsection{PCA in Time Series}


\section{Web Search Data in Financial Applications}





