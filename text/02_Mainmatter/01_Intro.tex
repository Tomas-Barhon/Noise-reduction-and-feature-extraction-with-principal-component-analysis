\chapter{Introduction}
\label{chap:one}

Since the introduction of the first cryptocurrency \ac{BTC}
associated with the unknown author Satoshi \cite{Nakamoto2008} cryptocurrencies
have become part of our everyday life. Their high volatility, futuristic name 
and alternative nature are of interest to the media and the general public.
According to \textbf{\href{https://coinmarketcap.com/charts/}{coinmarketcap.com}}
the overall cryptocurrency 
market capitalization peaked at around 3.71 trillion \$USD in the year 2024
which makes them a substantial part of the financial sphere.
The initial idea of \ac{BTC} was to establish an alternative to traditional fiat currencies. 
The \ac{BTC} whitepaper
pointed out the weakness of the current trust-based model that relies on a third-party instance responsible
for verifying transactions.
A different approach was suggested to validate transactions known as the proof-of-work which
utilizes the computational power of miners in the network. The fact that the power is 
distributed across the network ensures that it becomes exponentially harder with an increasing number of blocks
to generate blocks faster than the rest of the miners \citep[pg.~6]{Nakamoto2008}. 
The mining process is interconnected with the creation of new coins which is a crucial parameter
in all monetary systems. This fact gives researchers such as \cite{Kukacka2023} 
the possibility to use various attributes of the network to study the pricing dynamics of cryptocurrencies. 
However, there are a couple of substantial drawbacks that make price modeling relatively challenging.
Those are non-stationarity of the target prices, relatively short historical window, the limited power of
proxies for speculative components and as pointed out by many researchers 
such as \citet{Bouri2022}, \citet{Dimpfl2021} and \citet{Watorek2023} an idiosyncratic noise in volatility.
Addressing these issues might potentially lead to better-performing models, especially
with longer forecasting periods. Likewise in other fields, the recent rise of machine learning 
has also affected the cryptocurrency area where various \ac{ML} and \ac{DL} models 
are often being used 
to model the price \cite{Khedr2021} or volatility \cite{Kristjanpoller2018}. 


The main objective of this thesis is to try to tackle the problem of idiosyncratic noise in
the high dimensional data used for price and returns modeling across three ML models: Ridge \ac{LR}, \ac{SVM}
and \ac{LSTM} \ac{RNN}.
We will examine the effect of a method known as \acl{PCA} which was according to 
\cite{Farebrother2022} developed in 1933 by Harold Hotelling. However, others often refer
to the fact 
that the idea was already introduced before by Karl Pearson in the article 
\textit{On lines and planes of closest fit to systems of points in space} \citep{Pearson1901}.
This technique aims to compress data from a higher dimensionality space into a lower space while 
retaining a maximum amount of variance. It utilizes linear transformation of the covariance matrix
to do that.
Nevertheless, despite the initial focus on dimensionality reduction different types of 
\ac{PCA} are often being used as noise reduction techniques in signal or image processing.
Interestingly many studies in recent years have incorporated \ac{PCA} for time series data 
as a part of their preprocessing pipeline (\citet{Chowdhury2018}, \citet{Kristjanpoller2018}).
The idea stems from the fact that removing the most idiosyncratic components
might help with capturing clear dynamics that enter the price-making process.
We perceive that there is currently a lack of literature that would examine the effects of 
noise reduction techniques on the performance \ac{ML} based models
for cryptocurrency modelling. We want to mitigate most of the identified challenges using 
the currently available academic knowledge and focus exclusively on the effect of noise in the data. 
Admittedly it is always intricate to establish a ceteris paribus relationship in such a scenario
where many variables change, the randomness of the training process using \ac{SGD} plays a crucial role 
and the size of the dataset is relatively limited. We want to contribute with an alternative approach, especially 
in the preprocessing pipeline that can be used in future studies to decrease the volatility of predictions.
We do not aim to provide a universally applicable approach, as different techniques 
can produce varying outcomes on different datasets.
This phenomenon partially corresponds to the \textit{No Free Lunch Theorem} \citep{Wolpert1995} which
has turned into a buzzword in the \ac{ML} community over the years.


The remainder of the thesis is organized as follows: 
The following chapter introduces the fundamentals of cryptocurrencies and their unique characteristics.
It also covers the use of \ac{ML} methods in this field and especially focuses on the literature about
the use of \ac{PCA} in various areas. The data chapter explains in detail which data were used 
and elaborates on the basic resampling methods that we used. In methodology, we focus on each specific
\ac{ML} method and explain the core concepts that are crucial for understanding the training process. 
Similarly, we propose our complete forecasting framework. Chapter results and discussion evaluates
the findings for each currency-model pair across different settings. We also include a limitations
section which is especially crucial for our study where we acknowledge those problematic parts of our approach 
that might be improved in the future. The conclusion focuses on the overall impact and proposes
paths that should be explored in the years to come. All the tables, source codes and visualizations
can be found in the appendices.
