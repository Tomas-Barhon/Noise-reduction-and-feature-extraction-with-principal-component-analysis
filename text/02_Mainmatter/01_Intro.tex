\chapter{Introduction}
\label{chap:one}

Since the introduction of the first cryptocurrency \ac{BTC}
associated with the unknown author \cite{Nakamoto2008} cryptocurrencies
have become part of our everyday life. Their high volatility, futuristic name 
and alternative nature are of interest to the media and the general public.
According to on \textbf{\href{https://coinmarketcap.com/charts/}{coinmarketcap.com}}
the overall cryptocurrency market capitalization peaked at around 2.8 trillion \$USD in the year 2022
which makes them a substantial part of the financial sphere.
The initial idea of \ac{BTC} was to establish an alternative to traditional fiat currencies. 
The \ac{BTC} whitepaper
pointed out the weakness of the current trust-based model that relies on a third-party instance responsible
for verifying transactions.
A different approach was suggested to validate transactions known as the proof-of-work which
utilizes the computational power of miners in the network. The fact that the power is 
distributed across the network ensures that it becomes exponentially harder with an increasing number of blocks
to generate blocks faster than the rest of the miners \cite[pg.~6]{Nakamoto2008}. 
However, the mining process is interconnected with the creation of new \ac{BTC}s which is a crucial parameter
in all monetary systems. This fact gives researchers such as \cite{Kukacka2023} 
the possibility to use various attributes of the network to study the pricing dynamics of cryptocurrencies. 
On the other hand, there are a couple of substantial drawbacks that make price modeling relatively challenging.
Those are non-stationarity of the target prices, relatively short historical window, the limited power of
proxies for speculative components and as pointed out by many researchers 
such as \cite{Bouri2022}, \cite{Dimpfl2021}, \cite{Watorek2023} an idiosyncratic noise in volatility.
Addressing these issues might potentially lead to better-performing models, especially
with longer forecasting periods. Likewise in other fields, the recent rise of machine learning 
has also affected the cryptocurrency field where various \ac{ML} or \ac{DL} models are often being used 
to model the price \cite{Khedr2021} or volatility \cite{Kristjanpoller2018}. 


The main objective of this thesis is to try to tackle the problem of idiosyncratic noise in
the high dimensional data used for price and returns modeling.


