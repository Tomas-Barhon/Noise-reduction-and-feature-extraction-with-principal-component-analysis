\chapter{Conclusion}
\label{chap:six}

In this thesis,
we have studied the existing literature
on the use of machine learning in the field of cryptocurrencies
and other topics in cryptocurrency pricing and prediction.
Our aim was to provide a comprehensive overview of the current state
of the art in this field and further 
develop on the gaps identified in the literature.
We particularly focused on the use
of \ac{PCA} as a noise reduction technique. 

As cryptocurrencies are
a new and rapidly evolving field, there is a lack of
high quality literature wich leaves
a lot of room for improvement. 
The ability to predict cryptocurrency prices is of great interest
for investors and traders, as it can lead to significant profits.
Compared to traditional econometric analysis, 
our approach emphasizes the 
forecasting ability for out-of-sample data. Which 
is often overlooked in the literature.

We have iterated through many different model specifications 
described in the results section \ref{sec:results} and 
experimented with different cross-validation techniques.
We conclude that due to poor model performance,
we are unable to test whether the use of \ac{PCA} 
as a noise reduction
technique is beneficial for the prediction of cryptocurrency prices.
Our results show the shortcomings of
machine learning generalization in 
case of varying data distributions and
signal sparse environments.
We show why particular types of non-stationarity
can be problematic for time series cross-validation schemes.

Despite the lack of significant results,
we believe that we have made a 
strong argument 
for the reasons of 
out-of-sample performance deterioration.
We conclude that poor data quality
and the changing nature of the pricing dynamics 
are the main reasons for the poor model performance. We thus
suggest that future research should focus on improving the data quality
especially on the speculative side 
of the market.
We also suggest to focus on 
data augmentation techniques to improve
model generalization. 
