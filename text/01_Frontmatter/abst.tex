\section*{Abstract}

This thesis focuses on the use of machine learning
methods in the field of cryptocurrency forecasting. 
We forecast the price and log-returns of Bitcoin, Ethereum and Litecoin
using a variety of machine learning models with different complexity.
As time series
data and especially cryptocurrency data is often noisy, we aim to improve the	
forecasting ability by the use of noise reduction techniques particularly 
principal component analysis.
We found that the use of principal component analysis
does have little to no effect on the forecasting ability of the models
and that the models are not able to generalize well despite 
strong regularization. 
We analyze the reasons for this phenomenon and suggest that the
weekly granularity of speculative proxies 
and the changing nature of the pricing dynamics
are the main reasons for the poor model out-of-sample performance.


\bigskip

\begin{tabular}{lp{8.6cm}}
		\textbf{JEL Classification} & \JEL \\
		\textbf{Keywords} & \Keywords \\
 		& \\
		\textbf{Title} & \Bookname \\
 		\textbf{Author's e-mail} & \texttt{\href{mailto:\Email}{\Email}}\\
		\textbf{Supervisor's e-mail} & \texttt{\href{mailto:\EmailSup}{\EmailSup}}\\
\end{tabular}

\bigskip

\section*{Abstrakt}\label{abstract}
Tato práce se zaměřuje na použití metod strojového učení v oblasti
predikování cen a logaritmických výnosů Bitcoinu, Etherea a Litecoinu
za použítí několika modelů strojového učení s různou mírou komplexity.
Jelikož jsou časové řady a zejména u kryptoměn zašuměná, cílíme
na zlepšení predikčních schopností pomocí technik redukce šumu 
konkrétně pomocí analýzy hlavních komponent.
Naše práce ukazuje, že použití analýzy hlavních komponent
nemá téměř žádný vliv na predikční schopnosti modelů
a že modely nejsou schopny dobře generalizovat na budoucí data
i přes silnou regularizaci. Analyzujeme
důvody tohoto jevu a docházíme k závěru, že
týdení granularita spekulativních proxy proměnných a měnící se cenová dynamika 
jsou hlavními důvody špatné výkonnosti modelů na testovacích datech.

\bigskip

\begin{tabular}{lp{7.7cm}}
		\textbf{Klasifikace JEL} & \JEL \\
		\textbf{Klíčova slova} & \Klic \\
 		& \\
		\textbf{Název práce} & \BooknameCZ \\
 		\textbf{E-mail autora} & \texttt{\href{mailto:\Email}{\Email}}\\
		\textbf{E-mail vedoucího práce} & \texttt{\href{mailto:\EmailSup}{\EmailSup}}\\
\end{tabular}

